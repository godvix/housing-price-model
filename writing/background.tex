% !TeX root = main.tex
\section{研究背景}
房价问题一直以来都是百姓关注的热门话题.
作为国家的支柱产业之一, 房地产的走势常常能够影响到区域经济的发展.
近年来不断走高的房价, 不断膨胀的房地产泡沫令人担忧.
但是, 住房的真实价值却一直是一个未知数.
在不同人眼中, 住房的价值可能完全不同.
但对于消费市场而言, 住房的真实价值应当是相对确定的.
通过分析住房价格的影响因素能够一定程度上确定商品房的市场价格中金融属性所占的比重.

现有住房价格影响因素的分析往往基于居民收入, 税收政策等宏观因素, 或基于交通设施等单一因素.
住房价格通常与周边诸多环境, 基础设施等高度相关, 而不仅仅只与单一因素相关.
如果能够量化这种多元相关性, 住房价格的空间分布能够在一定程度上表征城市居民对住房周边设施所带来的效益的支付意愿, 这将能够作为评价支付意愿的重要依据之一, 有利于计算难以量化的外部性的影子价格.
此外, 通过引入更加全面的影响因素, 可能可以识别出城市的特征, 发现不同城市间的偏好差异.
