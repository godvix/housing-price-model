% !TeX root = main.tex
\section{引言}
房价问题一直以来都是百姓关注的热门话题.
作为国家的支柱产业之一, 房地产的走势常常能够影响到区域经济的发展.
近年来不断走高的房价, 不断膨胀的房地产泡沫令人担忧.
但是, 住房的真实价值却一直是一个未知数.
在不同人眼中, 住房的价值可能完全不同.
但对于消费市场而言, 住房的真实价值应当是相对确定的.
通过分析住房价格的影响因素能够一定程度上确定商品房的市场价格中金融属性所占的比重.

现有住房价格影响因素的分析往往基于居民收入, 税收政策等宏观因素, 或基于交通设施等单一因素.
住房价格通常与周边诸多环境, 基础设施等高度相关, 而不仅仅只与单一因素相关.
如果能够量化这种多元相关性, 住房价格的空间分布能够在一定程度上表征城市居民对住房周边设施所带来的效益的支付意愿, 这将能够作为评价支付意愿的重要依据之一, 有利于计算难以量化的外部性的影子价格.
此外, 通过引入更加全面的影响因素, 可能可以识别出城市的特征, 发现不同城市间的偏好差异.

在课堂上, 老师介绍了市场分析中的回归分析法.
回归分析法可以通过输入数据, 得到一系列自变量对于因变量的影响因子, 给出不同自变量对于因变量的影响程度大小;同时, 这种影响因子的结果也有助于预测一定自变量条件下因变量的值.

这样的模型可以通过考虑交通便捷性对于房价的影响来估算人们对于通勤时间的支付意愿, 从而计算地铁的社会总效益.
经过思考后与阅读文献后, 本文拟采用回归分析法来对房价的影响因素进行分析, 得到因素对于房价的作用因子;得到影响因子后, 本文借助该模型对于房价进行预测.
