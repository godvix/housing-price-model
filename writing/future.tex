% !TeX root = main.tex
\section{未来工作}
\subsection{模型选择}
受限于专业知识的不足, 笔者无法建立更加符合现实情况的模型进行统计.
目前人工智能发展迅速, 使用柔性更强的神经网络也许能够取得更好的结果.

此外, 由于北京, 上海的住房价格分布呈现明显的单中心性, 某地的住房价格很大程度上由其与市中心的距离决定, 这使得细部特征的研究较为困难.
如果能够使用统计学方法更加精确地提取细部特征, 则能够取得更好的效果.

除地理位置以外, 物业水平, 地方政策, 住房建成年限, 装修情况, 居民收入等也是影响房价的重要因素.
受限于数据来源的不足, 本研究暂未将其它因素纳入考虑范围.

\subsection{数据清洗}
本研究所选取的北京住房价格数据来源于 2011 -- 2017 年的链家数据, 时间跨度大且较为老旧.
因此模型拟合的效果欠佳.
使用与 POI 数据同期的较新数据也许能够取得令人更加满意的结果.
% 拟合结果显示最小的特征值仅有 \num{1.86e-24}.
% 这意味着自变量, 即 $distance_i$ 之间很可能有较强的相关性.
% 而事实上也确实如此, 区位中心往往集中分布于城市中心, 从较大的尺度来看, 与区位中心的距离约等于与城市中心的距离, 因而具有强相关性.
% 可以考虑使用 Principal Component Analysis (PCA) 对自变量进行降维.
